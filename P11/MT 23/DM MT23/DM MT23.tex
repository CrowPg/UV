\documentclass[11pt, fleqn]{report}
\usepackage[utf8]{inputenc}
\usepackage{amsmath}
\usepackage{amssymb}
\renewcommand \familydefault{cmss}
\usepackage[cm]{sfmath}
\begin{document}

\title{DM MT23 : Exercices 12 et 16 du Chp. 3}
\author{Salah Benmoussati, Julien Descamps, Robin Hérin, Johan Medioni}
\date {}
\maketitle

\noindent Exercice 12
\\
\\

\noindent 1. \, $A=\left(\begin{array}{ccc}
1 & 1 & 1\\
1 & j & j^2\\
1 & j^2 & j\\
\end{array}\right)$ \newline
\newline
Calcul de l'inverse par la méthode des cofacteurs :
\\
\\
$A=(a_{ij})_{(i,j)\in \{1,2,3\} }$
\\
\\
$ cof(a_{11}) = \left|\begin{array}{cc}j & j^2 \\j^2 & j\end{array}\right| = (j^2-j^4) = (j-j^2)(j^2+j)=j^2-j $ \\
$ cof(a_{12}) = - \left|\begin{array}{cc}1 & j^2 \\1 & j\end{array}\right| = -( j-j^2 ) = j^2-j $ \\
$ cof(a_{13}) = \left|\begin{array}{cc}1 & j \\1 & j^2\end{array}\right| = j^2-j$ \\
$ cof(a_{21}) = -\left|\begin{array}{cc}1 & 1 \\j^2 & j\end{array}\right| = j^2-j $ \\
$ cof(a_{31}) = \left|\begin{array}{cc}1 & 1 \\j & j^2\end{array}\right| = j^2-j $ \\
$ cof(a_{22}) = \left|\begin{array}{cc}1 & 1 \\1 & j\end{array}\right| = j-1 $ \\
$ cof(a_{23}) = -\left|\begin{array}{cc}1 & 1\\1 & j^2\end{array}\right| = 1-j^2 $ \\
$ cof(a_{32}) = -\left|\begin{array}{cc}1 & 1\\1 & j^2\end{array}\right| = 1-j^2 $ \\
$ cof(a_{33}) = \left|\begin{array}{cc}1 & 1 \\1 & j\end{array}\right| = j-1 $ \\
\\
\\
$ co(U) = \left(\begin{array}{ccc}j^2-j & j^2-j & j^2-j \\j^2-j & j-1 & 1-j^2 \\j^2-j & 1-j^2 & j-1\end{array}\right) $
\\
\\
\\

\begin{align*}
det(U) 
&= \left|\begin{array}{ccc}1 & 1 & 1 \\1 & j & j^2 \\1 & j^2 & j\end{array}\right| \\
&=  a_{11}cof(a11)+a_{12}cof(a12)+a_{13}cof(a13) \\
&= 3(j^2-1) \\
\end{align*}
\\
\\
\\

$U^{-1} = \dfrac {1} {det(U)} co(U) = \dfrac {1}{3} \dfrac {1}{j^2-1} \left(\begin{array}{ccc}j^2-j & j^2-j & j^2-j \\j^2-j & j-1 & 1-j^2 \\j^2-j & 1-j^2 & j-1\end{array}\right)= \dfrac {1}{3}\left(\begin{array}{ccc}1 & 1 & 1 \\1 & \frac{j-1}{j^2-1} & -1 \\1 & -1 & \frac{j-1}{j^2-1}\end{array}\right) $
\\
\\
\\

\noindent Exercice 16
\\
\\
\noindent 2. $ A=\left(\begin{array}{ccc}1 & -1 & -1 \\2 & 2 & -3 \\1 & 2 & -1 \\4 & 0 & -5\end{array}\right) $
\\
\\

\noindent $ 
\begin{array}{cc}
	\left(\begin{array}{ccc}
		1 & -1 & -1 \\
		2 & 2 & -3 \\
		1 & 2 & -1 \\
		4 & 0 & -5
		\end{array}\right) 
	& \begin{array}{l}
		L_1 \\
		L_2 \\
		L_3 \\
		L_4 \\
	\end{array}
\end{array}
$
\\
\\

\noindent $ 
\begin{array}{cc}
	\left(\begin{array}{ccc}
		1 & -1 & -1 \\
		0 & 4 & -1 \\
		0 & 3 & 0 \\
		0 & 4 & -1
		\end{array}\right) 
	& \begin{array}{l}
		L_1 \\
		L_2 \leftarrow L_2-2L_1 \\
		L_3\leftarrow L_3-L_1 \\
		L_4\leftarrow L_4-4L_1 \\
	\end{array}
\end{array} $
\\
\\

\noindent $
\begin{array}{cc}
	\left(\begin{array}{ccc}
		1 & -1 & -1 \\
		0 & 4 & -1 \\
		0 & 0 & 3 \\
		0 & 0 & 0
		\end{array}\right) 
	& \begin{array}{l}
		L_1 \\
		L_2 \\
		L_3\leftarrow 4L_3-3L_2  \\
		L_4\leftarrow L_4-L_2 \\
	\end{array}
\end{array} $

$ \begin{cases}
	x_1-x_2-x_3 = 0\\
	4x_2-x_3 = 0\\
	3x_3 = 0
\end{cases} $ Il vient $x_1=x_2=x_3=0.$
\\
\\
Avec $b=\left(\begin{array}{c}1 \\0 \\0 \\2\end{array}\right)$ :
\\
$\left(\begin{array}{c}1 \\-2 \\-1 \\-2\end{array}\right)$
\\
$\left(\begin{array}{c}1 \\-2 \\2 \\0\end{array}\right)$ Et le système qui s'en suit est :
\\
$ \begin{cases}
	x_1-x_2-x_3 = 1\\
	4x_2-x_3 = -2\\
	3x_3 = 2
\end{cases}
\Leftrightarrow
\begin{cases}
	x_1 = \frac{4}{3}\\
	x_2= -\frac{1}{3}\\
	x_3 = \frac{2}{3}
\end{cases} $
\\
\\
Avec $b=\left(\begin{array}{c}0 \\1 \\0 \\0\end{array}\right)$ :
\\
$\left(\begin{array}{c}0 \\1 \\0 \\0\end{array}\right)$
$\left(\begin{array}{c}0 \\1 \\-3 \\-1\end{array}\right)$ ce qui est impossible car $0\neq-1$. Pas de solution.
\\
\\
\\
\\
\noindent 3. $ A=\left(\begin{array}{cccc}1 & -1 & 1 & 1 \\1 & 1 & -1 & 2 \\0 & 1 & 2 & 0\end{array}\right) $
\\
\\

\noindent $
\begin{array}{cc}
	\left(\begin{array}{cccc}
		1 & -1 & 1 & 1 \\
		0 & 2 & -2 & 1 \\
		0 & 1 & 2 & 0
	\end{array}\right)
	& \begin{array}{l}
	L_1\\
	L_2 \leftarrow L_2-L_1\\
	L_3
	\end{array}
\end{array}	 $
\\
\\

\noindent $
\begin{array}{cc}
	\left(\begin{array}{cccc}
		1 & -1 & 1 & 1 \\
		0 & 1 & 2 & 0 \\
		0 & 2 & -2 & 1
	\end{array}\right)
	& \begin{array}{l}
	L_1\\
	L_2 \leftarrow L_3\\
	L_3 \leftarrow L_2
	\end{array}
\end{array}	 $
\\
\\

\noindent $
\begin{array}{cc}
	\left(\begin{array}{cccc}
		1 & -1 & 1 & 1 \\
		0 & 1 & 2 & 0 \\
		0 & 0 & -6 & 1
	\end{array}\right)
	& \begin{array}{l}
	L_1\\
	L_2 \\
	L_3 \leftarrow L_3-2L_2
	\end{array}
\end{array}	 $ 
\\
\\
$ \begin{cases}
	x_1-x_2+x_3+x_4 = 0 \\
	x_2+2x_3 = 0 \\
	-6x_3+x_4 = 0
\end{cases} 
\Leftrightarrow
\begin{cases}
	x_4 = 6x_3\\
	x_2 = -2x_3\\
	x_1 = -9x_3
\end{cases}$
\\
On a donc : $x=\alpha \left(\begin{array}{c} -9\\-2\\1\\6 \end{array}\right)$
\\
Avec $b=\left(\begin{array}{l}b_1 \\b_2 \\b_3\end{array}\right)$ : 
\\
$\left(\begin{array}{l}b_1 \\b_2-b_1 \\b_3\end{array}\right)$
\\
$\left(\begin{array}{l}b_1 \\b_3\\b_2-b_1\end{array}\right)$
\\
$\left(\begin{array}{l}b_1 \\b_3\\b_2-b_1-2b_3\end{array}\right)$
\\
$ \begin{cases}
	x_1-x_2+x_3+x_4 = b_1 \\
	x_2+2x_3 = b_3 \\
	-6x_3+x_4 = b_2-b_1-2b_3
\end{cases}
\Leftrightarrow
\begin{cases}
	x_4 = 6x_3 + b_2-b_1-2b_3\\
	x_2 = -2x_3 + b_3\\
	x_1 = -9x_3 +2b_1-b_2+3b_3
\end{cases}
$ 
\\
On a donc : $ x = \alpha \left(\begin{array}{c} -9\\-2\\1\\6 \end{array}\right) + \left(\begin{array}{l} 2b_1-b_2+3b_3 \\ b_3 \\
0 \\ -b_1 + b_2 - 2b_3 \end{array} \right) $
\\
\\
\\
\\ 
\noindent 4. $ A=\left(\begin{array}{cccc}1 & -1 & 1 & 1 \\1 & 1 & -1 & 2 \\0 & 2 & -2 & 1\end{array}\right) $
\\
\\
$ \begin{array}{cc}
	\left(\begin{array}{cccc}
		1 & -1 & 1 & 1 \\
		1 & 1 & -1 & 2 \\
		0 & 2 & -2 & 1
	\end{array}\right)
	&
	\begin{array}{l}
		L_1 \\
		L_2 \\
		L_3
	\end{array}	
\end{array} $
\\
\\
$ \begin{array}{cc}
	\left(\begin{array}{cccc}
		1 & -1 & 1 & 1 \\
		0 & 2 & -2 & 1 \\
		0 & 2 & -2 & 1\end{array}\right)
	&
	\begin{array}{l}
		L_1 \\
		L_2 \leftarrow L_2-L_1 \\
		L_3
	\end{array}	
\end{array} $		
\\
\\
$ \begin{array}{cc}
	\left(\begin{array}{cccc}
		1 & -1 & 1 & 1 \\
		0 & -2 & 2 & 1 \\
		0 & 0 & 0 & 0 
	\end{array}\right)
	&
	\begin{array}{l}
		L_1 \\
		L_2  \\
		L_3 \leftarrow L_3-L_2
	\end{array}	
\end{array} $	
\\
\\
$ \begin{cases}
	x_1-x_2+x_3+x_4=0\\
	-2x_2+2x_3+x_4=0\\
\end{cases} 
\Leftrightarrow
\begin{cases}
	x_4=2(x_2-x_3)\\
	x_1=x_2-x_3-2x_2+2x_3=x_3-x_2\\
\end{cases}$
\\
\\
On a donc : 
\begin{align*} 
	x
	&= (x_3-x_2 ; x_2 ; x_3 ; 2x_2-2x_3)\\
	&= x_2(-1;1;0;2) + x_3(1;0;1;-2)\\
	&= \alpha \left(\begin{array}{l} -1\\1\\0\\2 \end{array}\right) + \beta \left(\begin{array}{l} 1\\0\\1\\-2 \end{array} \right)
\end{align*} 
\\
\\
Avec $ b = \left(\begin{array}{c}0 \\2 \\2\end{array}\right) $ :
\\
\\
$ \left(\begin{array}{c}0 \\2 \\0\end{array}\right) $
\\
\\
$ \begin{cases}
	x_1-x_2+x_3+x_4=0\\
	-2x_2+2x_3+x_4=2\\
\end{cases} 
\Leftrightarrow
\begin{cases}
	x_4=2(x_2-x_3)+2\\
	x_1=x_2-x_3-2x_2+2x_3-2=x_3-x_2-2\\
\end{cases}$
\\
\\
On a donc $ x = \alpha \left(\begin{array}{l} -1\\1\\0\\2 \end{array}\right) + \beta \left(\begin{array}{l} 1\\0\\1\\-2 \end{array} \right) + \left(\begin{array}{c}1 \\0 \\0 \\1\end{array}\right) $
\\
\\
\\
Avec $ b = \left(\begin{array}{c}1 \\1 \\1\end{array}\right) $ :
\\
\\
$ \left(\begin{array}{c}1 \\0 \\1\end{array}\right)
\\
\\
\left(\begin{array}{c}1 \\0 \\1\end{array}\right) $ Ce qui est impossible car $ 0 \neq 1 $. Pas de solution.

	





\end{document}
